\documentclass[letterpaper]{article}
\usepackage[english]{babel}
\usepackage[utf8]{inputenc}
\usepackage[top=1.2in, left=0.9in, bottom=1.2in, right=0.9in]{geometry} % sets the margins
\usepackage{amsmath,amsthm,amssymb}
\usepackage{url} % fixes url problem
\usepackage{csquotes}% Recommended
\usepackage[doublespacing]{setspace} % turns on double spacing

\usepackage[style=authoryear-ibid,backend=biber]{biblatex}

\addbibresource{references.bib}% Syntax for version >= 1.2

\title{A simple reflection for the forum}
\author{Niloofar Khoda Bakhsh}
\date{Fall Quarter 2016}

\begin{document}
\maketitle
\begin{abstract}
The purpose of this paper is to reflect the forum sessions and speakers talks on multimedia, technology, art and business matters. While each speaker had their unique perspective through these subjects, they shared some similar work paths that I talk about in the essay.
\end{abstract}

parskip}Forum sessions were really informative and beneficial for me. As each speaker brought their unique opinion on art or technology, it was interesting to see different but not necessarily opposing opinions on this subject.
There were good tips to pick up from each session of the forum. Darnell Kamp made coming up with an idea easier for me. The steps he mentioned in pitching a startup and the red and blue ocean strategy that he mentioned was really helpful for me because I have never taken a business class.
In my view, Waffa and Sajid talk is a great example of collaboration and teamwork. Just like how easily they offered their help to us and were open to sharing their contact information. I think making connections should be this easy.
Rhonda’s, Kadet’s and Kal’s works, although don’t seem to be similar, they have a similarity. They all share the idea of human and technology. This might be just a matter of living in contemporary time. Maybe that’s issue of the time for all multimedia artists or maybe this similarity is the reason they have been chosen to talk in the forum. But what is interesting for me is that each artist, express their issues with their own language and the final results are far too different from each other.
Rhonda’s concerns appear in her military-themed works. She explores the world of humans and robots and questions if we are ready for what we can do with technology? Kal uses technology to express to express things that only humans understand: feelings, fears, and fun. His works are also great examples of interactivity. 
I believe interactivity matters in contemporary art because the contemporary audience is not the passive audience who just walks through museums and galleries and looks at art. Not that making the traditional form of art or following it, is not valuable but contemporary art should be interactive because the contemporary audience is interactive and need something that they can respond to.
All in all, listening to the professionals, is really profitable and after listening to five of them this quarter is not hard to say that in a way they all follow the same path. They all research, take risks, spend all their energy and time into their works. They use any medium that helps them execute their ideas. In this way, they either get to learn something new or get help from people who know that medium. Almost all of the speakers talked about being your true self and convey what you think is true. They all mentioned the importance of making connections and giving back to the community/people. These are all points that I take away from this class and hope to be able to apply to my own path to multimedia art in the future.
\end{document}

